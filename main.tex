\documentclass[11pt, a4paper]{article}

% Packages:
\usepackage[
    ignoreheadfoot, % set margins without considering header and footer
    top=1.2 cm, % seperation between body and page edge from the top
    bottom=1.2 cm, % seperation between body and page edge from the bottom
    left=1 cm, % seperation between body and page edge from the left
    right=1 cm, % seperation between body and page edge from the right
    % showframe % for debugging 
]{geometry} % for adjusting page geometry
\usepackage{titlesec} % for customizing section titles
\usepackage{tabularx} % for making tables with fixed width columns
\usepackage{array} % tabularx requires this
\usepackage[dvipsnames]{xcolor} % for coloring text
\definecolor{primaryColor}{RGB}{0, 0, 255} % blue color % define primary color
\usepackage{enumitem} % for customizing lists
\usepackage{fontawesome5} % for using icons
\usepackage{amsmath} % for math
\usepackage[
    pdftitle={Dhruv Khatri's CV},
    pdfauthor={Dhruv Khatri},
    pdfcreator={Dhruv},
    colorlinks=true,
    urlcolor=primaryColor
]{hyperref} % for links, metadata and bookmarks
\usepackage[pscoord]{eso-pic} % for floating text on the page
\usepackage{calc} % for calculating lengths
\usepackage{bookmark} % for bookmarks
\usepackage{lastpage} % for getting the total number of pages
\usepackage{changepage} % for one column entries (adjustwidth environment)
\usepackage{paracol} % for two and three column entries
\usepackage{ifthen} % for conditional statements
\usepackage{needspace} % for avoiding page brake right after the section title
\usepackage{iftex} % check if engine is pdflatex, xetex or luatex

\usepackage{ragged2e}

% Ensure that generated pdf is machine readable/ATS parsable:
\ifPDFTeX
    \input{glyphtounicode}
    \pdfgentounicode=1
    \usepackage[T1]{fontenc}
    \usepackage[utf8]{inputenc}
    \usepackage{newtxtext,newtxmath} % Times New Roman text + math
\fi

% Some settings:
\linespread{1.10}\selectfont
\raggedright
\AtBeginEnvironment{adjustwidth}{\partopsep0pt} % remove space before adjustwidth environment
\pagestyle{empty} % no header or footer
\setcounter{secnumdepth}{0} % no section numbering
\setlength{\parindent}{0pt} % no indentation
\setlength{\topskip}{0pt} % no top skip
\setlength{\columnsep}{0.15cm} % set column seperation
\pagenumbering{gobble} % no page numbering

\titleformat{\section}{\needspace{4\baselineskip}\bfseries\large}{}{0pt}{}[\vspace{1pt}\titlerule]

\titlespacing{\section}{
    -1pt % left space
}{
    0.3 cm % top space
}{
    0.2 cm % bottom space
} % section title spacing

\renewcommand\labelitemi{$\vcenter{\hbox{\small$\bullet$}}$} % custom bullet points
\newenvironment{highlights}{
    \begin{itemize}[
        topsep=0.10 cm,
        parsep=0.10 cm,
        partopsep=0pt,
        itemsep=0pt,
        leftmargin=0 cm + 10pt
    ]
}{
    \end{itemize}
} % new environment for highlights

\newenvironment{highlightsforbulletentries}{
    \begin{itemize}[
        topsep=0.10 cm,
        parsep=0.10 cm,
        partopsep=0pt,
        itemsep=0pt,
        leftmargin=10pt
    ]
}{
    \end{itemize}
} % new environment for highlights for bullet entries

\newenvironment{onecolentry}{
    \begin{adjustwidth}{
        0 cm + 0.00001 cm
    }{
        0 cm + 0.00001 cm
    }
}{
    \end{adjustwidth}
} % new environment for one column entries

\newenvironment{twocolentry}[2][]{
    \onecolentry
    \def\secondColumn{#2}
    \setcolumnwidth{\fill, 4.5 cm}
    \begin{paracol}{2}
}{
    \switchcolumn \raggedleft \secondColumn
    \end{paracol}
    \endonecolentry
} % new environment for two column entries

\newenvironment{threecolentry}[3][]{
    \onecolentry
    \def\thirdColumn{#3}
    \setcolumnwidth{, \fill, 4.5 cm}
    \begin{paracol}{3}
    {\raggedright #2} \switchcolumn
}{
    \switchcolumn \raggedleft \thirdColumn
    \end{paracol}
    \endonecolentry
} % new environment for three column entries

\newenvironment{header}{
    \setlength{\topsep}{0pt}\par\kern\topsep\centering\linespread{1.5}
}{
    \par\kern\topsep
} % new environment for the header

\newcommand{\placelastupdatedtext}{%
  \AddToShipoutPictureFG*{%
    \put(
        \LenToUnit{\paperwidth-2 cm-0 cm+0.05cm},
        \LenToUnit{\paperheight-1.0 cm}
    ){\vtop{{\null}\makebox[0pt][c]{
        \small\color{gray}\textit{Last updated in September 2024}\hspace{\widthof{Last updated in September 2024}}
    }}}%
  }%
}%

% save the original href command in a new command:
\let\hrefWithoutArrow\href

\begin{document}

\newcommand{\AND}{\unskip
    \cleaders\copy\ANDbox\hskip\wd\ANDbox
    \ignorespaces
}
\newsavebox\ANDbox
\sbox\ANDbox{$|$}

\begin{header}
    \fontsize{18pt}{18pt}\selectfont \textbf{Dhruv Khatri}

    \vspace{5pt}

    \normalsize
    New Delhi, India
    \kern 5pt \AND \kern 5pt
    \hrefWithoutArrow{mailto:dhruvkhatri1234@gmail.com}{ dhruvkhatri1234@gmail.com}
    \kern 5pt \AND \kern 5pt
    \hrefWithoutArrow{tel:+91-981-889-63-02}{ +91-981-889-63-02}
    \kern 5pt \AND \kern 5pt
    \hrefWithoutArrow{https://leetcode.com/u/dhruvicous/}{Leetcode}
    \kern 5pt \AND \kern 5pt
    \hrefWithoutArrow{https://github.com/dhruvicious}{GitHub}
    \kern 5pt \AND \kern 5pt
    \hrefWithoutArrow{https://www.linkedin.com/in/thehalfbldprinc3/}{LinkedIn}
\end{header}

% ---------------------------- Education ----------------------------
\section*{Education}

\textbf{Bhagwan Parshuram Institute of Technology} \hfill Sept 2022 – June 2026 \\
\textit{Bachelor of Technology in Electronics and Communication Engineering} \hfill \textbf{CGPA: 6.552} \\
\textbf{Relevant Coursework:} Computer Networks, Machine Learning, Internet of Things, Mobile Computing, Control Systems

\vspace{0.15cm}

\textbf{Sidhhartha Public School} \hfill 2020 – 2022 \\
\textit{Senior Secondary (Class XII), CBSE} \hfill \textbf{Percentage: 87.8\%}

\vspace{0.15cm}

\textbf{Guru Harkrishan Public School, Hemkunt Colony} \hfill 2018 – 2020 \\
\textit{Secondary (Class X), CBSE}\hfill \textbf{Percentage: 76\%}
% ---------------------------- Skills ----------------------------

\vspace{0.1cm}
\section{Skills}

\begin{onecolentry}
    \textbf{Languages \& Frameworks:} C++, Python, JavaScript, TypeScript, React, Next.js, Node.js, Express, Tailwind CSS \\
    \textbf{Databases \& APIs:} PostgreSQL, MongoDB, REST, tRPC \\
    \textbf{Tools \& Platforms:} Git, GitHub, VS Code, Linux, macOS, Raspberry Pi \\
    \textbf{ML \& Data Science:} Pandas, NumPy, Scikit-learn, XGBoost, Streamlit
\end{onecolentry}

% ---------------------------- Projects ----------------------------

\vspace{0.1cm}
\section{Projects}

\textbf{Smart Mirror} \hfill \textit{Embedded \& Frontend-Focused Full-Stack} \\
\textit{Tech Stack: Raspberry Pi, Next.js, Web Speech API, Google Gemini API} \\
\textit{GitHub: \href{https://github.com/dhruvicious/smartMirror}{smartMirror}}

\vspace{0.1cm}
\begin{onecolentry}
    \begin{highlights}
        \item Reduced boot time by \textasciitilde40\%, achieving sub-20s startup on Raspberry Pi.
        \item Delivered real-time weather, calendar, and news widgets in Next.js with 1-min auto-refresh.
        \item Enabled 10+ voice commands (e.g., “show news”) via Web Speech API; Gemini API replies in <500ms.
        \item Cut memory usage by \textasciitilde35\% vs MagicMirror² using tree-shaking and optimized bundling.
    \end{highlights}
\end{onecolentry}

\vspace{0.3cm}
\textbf{Snake Game} \hfill \textit{Software Engineer (C++ / Raylib)} \\
\textit{Tech Stack: C++, Raylib, CMake, Catch2, Doxygen} \\
\textit{GitHub: \href{https://github.com/dhruvicious/SnakeGame}{SnakeGame}}

\vspace{0.1cm}

\begin{onecolentry}
    \begin{highlights}
        \item Engineered a modular Snake game with \textbf{layered architecture} (core logic, render, game loop).
        \item Applied \textbf{OOP patterns} (Game Loop, Command) and \textbf{S.O.L.I.D. principles} for extensibility.
        \item Integrated \textbf{unit testing (Catch2)}, automated builds (CMake), and documentation (Doxygen).
        \item Enhanced gameplay with \textbf{moving food mechanics}, multi-level layouts, and responsive UI.
    \end{highlights}
\end{onecolentry}

\vspace{0.3cm}
\textbf{Calories Burnt Prediction using XGBoost} \hfill \textit{Machine Learning Developer} \\
\textit{Tech Stack: Python, XGBoost, Pandas, NumPy, Scikit-learn, Streamlit} \\
\textit{GitHub: \href{https://github.com/dhruvicious/CalorieBurntPrediction}{CalorieBurntPrediction}}

\vspace{0.1cm}
\begin{onecolentry}
    \begin{highlights}
        \item Trained XGBoost on \textasciitilde15k exercise logs to predict calorie burn with \textasciitilde2.7 kcal MAE.
        \item Applied full preprocessing (encoding, outlier removal, normalization) for generalization.
        \item Deployed Streamlit app for real-time calorie estimates from live input in <1s.
    \end{highlights}
\end{onecolentry}

% ---------------------------- Achievements ----------------------------
\vspace{0.1cm}
\section{Achievements}

\begin{onecolentry}
    \begin{highlights}
        \item \textbf{Adobe India Hackathon (2025)} — Participated in nationwide hackathon.
        \item \textbf{Drone Project (2025)} — Restored quadcopter flight via hardware fixes and calibration.
        \item \textbf{Hardware Hackathon (2023)} — Built Smart Mirror prototype with Raspberry Pi in 24 hours.
        \item \textbf{Game Jam (2023)} — Developed 2D platformer with 5+ levels in 24 hours.
    \end{highlights}
\end{onecolentry}

\end{document}